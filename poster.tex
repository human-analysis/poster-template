\documentclass[final,red]{beamer}
\mode<presentation>
{
\usetheme{Icy}
}

\setbeamerfont{itemize}{size=\normalsize}
\setbeamerfont{itemize/enumerate body}{size=\normalsize}
\setbeamerfont{itemize/enumerate subbody}{size=\normalsize}

\usepackage{todonotes}
\usepackage{algorithm,algorithmic}
\usepackage{multirow}
\usepackage{times}
\usepackage{amsmath,amsthm, amssymb, latexsym}
\usepackage{mathtools}
\usepackage{exscale}
\usepackage{subcaption}
\usepackage{tikz}
\usepackage{sfmath}
\usepackage{xspace}
\usepackage{booktabs, array}
\usepackage[english]{babel}
\usepackage[latin1]{inputenc}
\usepackage{listings}
\usepackage[orientation=landscape,size=custom,width=240,height=120,scale=1.9]{beamerposter}
% \listfiles
% Display a grid to help align images
% \beamertemplategridbackground[1cm]

\usetikzlibrary{arrows,shapes,chains,matrix,backgrounds}
\usetikzlibrary{positioning,scopes,shadows,decorations}
\title{\Huge The World's Most Awesome Poster}

\author{Vishnu Naresh Boddeti \\ Michigan State University}
\institute[Michigan State University]{Michigan State University, East Lansing, MI, USA}
\date[\today]{\today}

\begin{document}

\begin{frame}{}
\begin{columns}[t]

\begin{column}{.3\linewidth}
\begin{block}{Object Alignment}
\begin{itemize}
		\item blah blah
		\item blah blah
    \vspace{1cm}
    \begin{figure}
    \centering
    \begin{subfigure}{0.9\textwidth}
		\missingfigure{missing figure}    	
    \end{subfigure}
    \hspace{2cm}
    \begin{subfigure}{0.9\textwidth}
    	\missingfigure{missing figure}
    \end{subfigure}
    
    \end{figure}
    \vspace{1cm}
    \item blah blah
    \item blah blah
	\end{itemize}
  \vspace{1cm}
  \begin{exampleblock}{{\large Contribution:}}
  \end{exampleblock}
  \vspace{-1cm}
  \begin{itemize}
  \item blah blah
  \item blah blah
  \item blah blah
  \end{itemize}
\end{block}

\begin{block}{Landmark Shape Model}
\vspace{-1cm}
\begin{exampleblock}{Bayesian Partial Shape Inference}
\end{exampleblock}

\begin{columns}
\begin{column}{0.4\linewidth}
\begin{figure}
	\centering
	\missingfigure{missing figure}
\end{figure}
\end{column}

\tikzstyle{format} = [draw, thin, fill=blue!20]
  \tikzstyle{medium} = [ellipse, , draw, thin, fill=green!20, minimum
    height=2.0em]
  \tikzstyle{decision} = [diamond, draw, fill=blue!20,
    text width=4.5em, text badly centered, node distance=2cm, inner sep=0pt]
  \tikzstyle{block} = [rectangle, draw, fill=blue!20,
    text width=3.0em, text centered, rounded corners, minimum height=3em]
  \tikzstyle{line} = [draw, -latex']
  \tikzstyle{cloud} = [draw, ellipse,fill=red!20, node
    distance=2cm,minimum height=2em]
  \tikzstyle{circ} = [draw, circle, fill=blue!20, node
    distance=2cm,minimum height=2em,drop shadow]
\begin{column}{0.4\linewidth}    
\begin{center}
\begin{figure}
\begin{tikzpicture}
\node[circ] (d1) {{\tiny $s_1$}};
\node[circ,above of=d1,node distance=5.5cm] (d2) {{\tiny $s_2$}};
\node[circ,right of=d2,node distance=5.5cm] (d3) {{\tiny $s_3$}};
\node[circ,above right of=d3,node distance=5.5cm] (d4) {{\tiny $s_4$}};
\node[circ,right of=d4,node distance=5.5cm] (d5) {{\tiny $s_5$}};
\node[circ,below right of=d5,node distance=5.5cm] (d6) {{\tiny $s_6$}};
\node[circ,right of=d6,node distance=5.5cm] (d7) {{\tiny $s_7$}};
\node[circ,below of=d7,node distance=5.5cm] (d8) {{\tiny $s_8$}};
\node[circ,left of=d8,node distance=5.6cm] (d9) {{\tiny $s_9$}};
\node[circ,left of=d9,node distance=13.0cm] (d10) {{\tiny $s_{10}$}};
\draw[-] (d1) -- (d2);
\draw[-] (d2) -- (d3);
\draw[-] (d3) -- (d4);
\draw[-] (d4) -- (d5);
\draw[-] (d5) -- (d6);
\draw[-] (d6) -- (d7);
\draw[-] (d7) -- (d8);
\draw[-] (d8) -- (d9);
\draw[-] (d9) -- (d10);
\draw[-] (d1) -- (d10);
\end{tikzpicture}
\end{figure}
\end{center}
\end{column}
\end{columns}

\vspace{1cm}
\begin{center}
      $ BPSI =  \begin{cases} S = \Phi b + \mu + \epsilon \\ Y_p = M_p(sRS+t+\eta) \\ Y_h = M_h(sRS+t)\end{cases}$
\end{center}
{\small $S$ - canonical shape, $Y_p$ - partial shape, $Y_h$ - hallucinated shape, $\Theta=\{s,R,t\}$, $M$ - Occlusion Mask}

\begin{exampleblock}{Occlusion Handling}
\end{exampleblock}
  \vspace{-2cm}
  {\tiny
	\begin{itemize}
		\item {\small Deformations lie on low-dimensional subspace, can estimate shape from partial observations.}
    \item {\small Sample and evaluate multiple unoccluded landmark subset ($k$ out of $N$) hypothesis.}
	\end{itemize}}

\end{block}
\end{column}

\begin{column}{.3\linewidth}
\begin{block}{Landmark Appearance Model}

\begin{exampleblock}{Local Feature Representation: HOG}
\end{exampleblock}
\end{block}
\end{column}

\begin{column}{0.3\linewidth}
\begin{block}{Databases and Experiments}
\begin{exampleblock}{Dataset}
\end{exampleblock}

\vspace{-1cm}
\begin{itemize}
\item Cars from 3500 images from MIT Street Scene dataset.
\item 3433 cars manually annotated with landmarks.
\item Preprocessed via Generalized Procrustes Analysis.
\end{itemize}
\end{block}

\begin{block}{References}
\end{block}
\begin{thebibliography}{9}
\bibitem{gentry} 
C. Gentry. \textit{A fully homomorphic encryption scheme}. Stanford University, 2009.
 
\bibitem{fan} 
J. Fan and F. Vercauteren.
\textit{Somewhat practical fully homomorphic encryption}.
IACR Cryptology ePrint Archive, 2012:144, 2012.
\end{thebibliography}

\end{column}
\end{columns}
\vfill
\end{frame}
\end{document}